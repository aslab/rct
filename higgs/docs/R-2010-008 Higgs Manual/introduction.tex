
\section{Document Purpose}

The ASys search for universal cognitive architectures generalises insights
gained in the study of animal brains, providing a model of cognition that effectively and
homogeneously integrates autonomic, emotional and cogniitive aspects.

This document describes the Higgs robot, part of the ASys Robot Control Testbed
(RCT) that has been selected as the target demonstration of the higher level cognitive dimensions into a physical
system. The SOUL cognitive architecture intends the exploitation of croscutting
design patterns to realise custom architectures targetted to specific uses.
While SOUL leverages detailed knowledge about mammal brains as of integration of
cognitive, emotional and autonomic aspects its main objective is to transcend the concrete brain
implementations improving engineering
capability for the construction of technical systems.

This implementation shall serve as vehicle for evaluating both the architecture
and the componental technologies generalising the components extracted from
biological models ---e.g. rat brain parts like the amygdala, the basal ganglia
or the hippocampus--- that are addressed in other research activities.

\section{Content}

This report describes the Higgs Robot Control Testbed.

\Figure{Higgs-01.png}{10cm}{The Higgs robot is the central objective of
development of the RCT, where the architecture for robust autonomy
is demonstrated.}{fig:higgs}

\section{Intended Audience}

This document is intended to provide a description of the concrete testbed for
all stakeholders involved in development and evaluation of the SOUL
architecture.

\section{Mandatory References}

The ASys project is the context for this work:

\begin{description}
\item [ASys Vision] includes a description of the long-term objectives of the research.
\item [OASys] contains the ontology under development for ASys.  
\end{description}

\section{Structure of the report}

This document purpose is to describe the implementation of a physical and
software asset. It consists of several chapters roughly organised in four
parts. Part 1 is dedicated to introductory materials that are useful to frame
the main content of the report. Part 2 describes the hardware. Part 3 is the
main content with the software and modelling description of the integrated architecture. 
Finally, Part 4 contains some appendices.

The content is as follows:

\begin{description}
  \item [Chapter 1] provides an overall description of the document
  and the context for this work. 
  \item [Chapter 2] contains the specification of the robot hardware.
  \item [Chapter 3] Describes the robot control software.
  \item [Chapter 4] contains an analysis of the mission.
\end{description}

\section{Format}

In the different chapters and section of this documents several systems or
devices are described, so as to make it perfectly clear, the description of
each one uses a tabular structure with the following formats: \\[2mm]
This is mandatory text\\ \texttt{this depends of each device}\\ \emph{this is a comment}.


\section{Acknowledgements}

This work has been partially supported by a European Commission grant to the
project ``Integrating Cognition, Emotion and Autonomy'' (ICEA IST-027819) and
by the ``Conscious Cognitive Control'' grant of the Ministry for Science and
Innovation.

