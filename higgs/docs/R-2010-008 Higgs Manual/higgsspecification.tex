
The RCT testbed is a mobile robotic system intended as platform to test the
technology developed in ICEA. The RCT platform is a pioneer mobile robot for
research that is specially designed for outdoors applications. The aim of the
RCT testbed is to purvey the system with the neccessary cognitive abilities so
as to fulfil complex task. This abstract aim is grounded in the specification of a high level mission and the evaluation of the OM consciousness architecture as to enhenace mission level robustness.

In this document the overall situation of the RCT platform of the ICEA Project
is presented. The purpose of the document is double:

\begin{itemize}
  \item Report the current state of the platform.
  \item Provide a short description of the platform.
\end{itemize}

The specification section is presented in the form of tables of specifications to capture the
main characteristics of the different devices that compose the RCT platform. For
much more detailed information on the RCT platform please refer to Higg's Manual ?poner enlace?






%------------------------------------------------------------------------------

	
	

\section{Base Platform: Pioneer 2-AT8}

%------------------------------------------------------------------------------
\begin{table}
\centering
\hspace{-4cm}
\begin{tabular}{|l|l|}
\hline

\textbf{RCT Identifier}: & RCT-Higgs \\
\hline

\textbf{RCT Name}: & Higgs \\
\hline

\textbf{Description}: & mobile robot for outdoors use \\
\hline

\textbf{Functionality role}: & Robot base \\
\hline

\textbf{Model}: & Pioneer 2AT-8 \\
\hline

\textbf{State}: & fully operative \\
\hline

\textbf{Serial number}: & AT8CDBB1722 \\
\hline

\textbf{Purchase date}: & 200302 \\
\hline

\textbf{Physical Characteristics}: & 
\parbox{8cm}{
	\begin{tabular}{l l}
    	\multicolumn{2}{l}{Alluminium body}\\
		\textbf{Dimensions} & 50x49x24 cm\\	
		\textbf{Weight} & 15 kg\footnote{one battery onboard}\\	
		\textbf{Payload} & 40 kg\\		
	\end{tabular}
}
 \\
\hline

\textbf{Physical links}: & 
\parbox{8cm}{
	\begin{tabular}{l l}
	\textbf{} & \\
	
	\textbf{} & \\
	
	\textbf{} & \\
		
	\end{tabular}
}

 \\
\hline

\textbf{Informational links}: & 
\parbox{8cm}{
	\begin{tabular}{l l}
	\textbf{In} & 5 analog\\
	
	\textbf{Out} & 2 analog\\
	
	\textbf{In\&Out} & 3 RS-232 ports, 8 bits bus\\
		
	\end{tabular}
}

 \\
\hline

\textbf{Power}: &  3 batteries 12VDC 7A-h (252W-h) \\
\hline

\textbf{Features}: & 

\parbox{10cm}{
	\begin{tabular}{l p{6.5cm}}
	\textbf{Microcontroller} & Hitachi H8S\\
	\textbf{Actuators:}	& \textbf{Skid steering:} \small{4 pittman motors
	GM9236E204}\\
	\textbf{Sensors:}	& \textbf{Sonar:} \small{2 arrays (front\&rear)  each of 8
	sensors, 20�}  \\
	& \textbf{Encoders:} \small{1 per wheel-- 34,000 conts/rev}\\
	& \textbf{Bumpers:} \small{5 front and 5 rear}\\
	\end{tabular}
	\\
	}
	\\

\textbf{Tech. doc.}: & \\
\hline

\end{tabular}
\end{table}
%-------------------------------------------------------------------------------


\section{Onboard Systems}

\subsection{Gene-6330}

%------------------------------------------------------------------------------
\begin{table}[ht!]
\centering
\hspace{-4cm}
\begin{tabular}{|l|l|}
\hline

\textbf{RCT Identifier}: &  RCT-Gene\\
\hline

\textbf{RCT Name}: &  placa Gene\\
\hline

\textbf{Description}: &  This is the onboard computer\\
\hline

\textbf{Functionality role}: & On board computer \\
\hline

\textbf{Model}: &  GENE-6330\\
\hline

\textbf{State}: &  Obsolete and retired\\
\hline

\textbf{Serial number}: &  \\
\hline

\textbf{Purchase date}: &  \\
\hline

\textbf{Physical Characteristics}: & 
\parbox{8cm}{
	\begin{tabular}{l l}
    	\textbf{Dimensions} & 146x101.6x26mm mm\\	
		\textbf{Weight} & 0.4 kg
	\end{tabular}
}\\
\hline

\textbf{Physical links}: & 
\parbox{8cm}{
	\begin{tabular}{l l}
		\textbf{} & attached inside Higgs\\			
	\end{tabular}
}\\
\hline

\textbf{Informational links}: &
\parbox{10cm}{
	\begin{tabular}{l p{7.5cm}}
		1 ethernet (RJ-45)	&	not used \\
		1 configurable port	RS-232/422/485 & not used\\
		1 port RS-232 (need to put a conector) & not used\\
		Wifi (purveyed by Compact PCMCIA) & main external interface\\
	\end{tabular}
}
\\
\hline

\textbf{Power}: &  typical 0.7W / Max 6.4W (+5V and +12V AT) from higgs
batteries\\
\hline

\textbf{Features}: & 
\parbox{10cm}{
	\begin{tabular}{l p{6cm}}
		\textbf{Microprocessor} & Transmeta Crusoe TM5400, 600MHz\\
		\textbf{Memory} & 64MB SDRAM + 256MB SDRAM\\
		\textbf{Flash Memory} & CompactFlash I type 4GB\\
		\textbf{Wifi} & purveyed by a Compaq WL110 PCMCIA + antenna\\
	\end{tabular}
}\\
\hline

\textbf{Tech. doc.}: & \\
\hline

\end{tabular}
\end{table}
%-------------------------------------------------------------------------------







\newpage
\subsection{Laptop}
%------------------------------------------------------------------------------
\begin{table}[ht!]
\centering
\hspace{-4cm}
\begin{tabular}{|l|l|}
\hline

\textbf{RCT Identifier}: &  RCT-laptop\\
\hline

\textbf{RCT Name}: &  vaio\\
\hline

\textbf{Description}: &  a lightweith high functional laptop\\
\hline

\textbf{Functionality role}: & On board computer \\
\hline

\textbf{Model}: & Sony Vaio TX2HP \\
\hline

\textbf{State}: &  Fully operative \\
\hline

\textbf{Serial number}: &  \\
\hline


\textbf{Purchase date}: &  \\
\hline

\textbf{Physical Characteristics}: & 
\parbox{8cm}{
	\begin{tabular}{l l}
		\textbf{Dimensions} & \\	
		\textbf{Weight} & \\	
	\end{tabular}
}\\
\hline

\textbf{Physical links}: & The Vaio is attached to the back part of Higgs' top \\
\hline

\textbf{Informational links}: & 
\parbox{8cm}{
	\begin{tabular}{l p{8cm}}
		\textbf{In\&Out} &  2 USB 2.0, wifi IEEE 802.11b/g, 1 ethernet, Bluetooth
		2.0\\
	\end{tabular}
}\\
\hline

\textbf{Power}: &  from its internal battery (7.5 hour)\\
\hline

\textbf{Features}: & 
\parbox{10cm}{
	\begin{tabular}{l p{6.5cm}}
		\textbf{Processor} & Intel Pentium M 1.1GHz\\
		\textbf{Memory} & 510 MB DDR2 SDRAM\\
		\textbf{OS} & Windows XP and Fedora ??\\
		\textbf{Hard drive} & 80 GB -4200 rpm\\
		\textbf{Display} & 11.1'' LCD\\
		\textbf{DVD+RW} & \\
	\end{tabular}
}\\
\hline

\textbf{Tech. doc.}: & \\
\hline

\end{tabular}
\end{table}
%-------------------------------------------------------------------------------

\newpage
\subsection{Arduino}
%-------------------------------------------------------------------------------
\begin{table}[ht!]
\centering
\hspace{-4cm}
\begin{tabular}{|l|p{10cm}|} 
\hline

\textbf{RCT Identifier}: &  RCT-arduino\\
\hline

\textbf{RCT Name}: &  arduino\\
\hline

\textbf{Description}: &  a simple board to conect some simple devices\\
\hline

\textbf{Functionality role}: & I/O board \\
\hline

\textbf{Model}: &  \\
\hline

\textbf{State}: &  Fully integrated \\
\hline

\textbf{Serial number}: &  \\
\hline

\textbf{Purchase date}: &  2007\\
\hline

\textbf{Physical Characteristics}: &
\parbox{8cm}{
	\begin{tabular}{l l}
		\textbf{Dimensions} & 175x55x50 \\
	\end{tabular}
}\\
\hline

\textbf{Physical links}: & Attached to the left side of the laser support\\
\hline

\textbf{Informational links}: &
\parbox{8cm}{
	\begin{tabular}{l l}
		USB & connected to laptop\\
		3-wire & connected to various sensors\\
	\end{tabular}
}\\
\hline

\textbf{Power}: & 6-12 VDC \\
\hline

\textbf{Features}: &
\parbox{10cm}{
	\begin{tabular}{l p{6.5cm}}
		\textbf{Sensors} & 2 accelerometers, 1 compass, 2 current and voltage sensors,
				1 servo, 1 potentiometer, 2 switches.
	\end{tabular}
}\\
\hline

\textbf{Tech. doc.}: & \\
\hline

\end{tabular}
\end{table}
%-------------------------------------------------------------------------------




\subsection{Laser}

%------------------------------------------------------------------------------
\begin{table}[ht!]
\centering
\hspace{-4cm}
\begin{tabular}{|l|l|}
\hline

\textbf{RCT Identifier}: &  RCT-Laser\\
\hline

\textbf{RCT Name}: &  Laser\\
\hline

\textbf{Description}: &  laser scanner for mobile robotics applications, SLAM,
navigation, etc\\
\hline

\textbf{Functionality role}: & Scan device \\
\hline

\textbf{Model}: &  Sick LMS-200\\
\hline

\textbf{State}: &  Integrated\\
\hline

\textbf{Serial number}: &  \\
\hline

\textbf{Purchase date}: &  \\
\hline

\textbf{Physical Characteristics}: & 
\parbox{8cm}{
	\begin{tabular}{l l}
		\textbf{Dimensions} & 155x156x265mm\\	
		\textbf{Weight} & 4.5kg\footnote{without the support}\\	
	\end{tabular}
}\\
\hline

\textbf{Physical links}:
& The laser support is screwed to the front part of Higgs' top. \\
& Includes a mechanism for tilting the laser by means of a servo.\\
& The RCT-Wrist is screwed to the top of the laser support
\\
\hline

\textbf{Informational links}: & 
\parbox{8cm}{
	\begin{tabular}{l l}
		RS-232/\textbf{422} & connected to vaio through USB/RS-232 converter\\	
	\end{tabular}
}\\
\hline

\textbf{Power}: & 24VDC 40 W\\
\hline

\textbf{Features}: & 
\parbox{10cm}{
\begin{tabular}{l p{5cm}}
  \textbf{Scanning angle} & 180� \\
  \textbf{Res\/typ. meas. accuracy} & 10mm$/\pm$35mm  \\
  \textbf{Accesories} & Metacrilate support $\pm$45� tilt with a Futaba servo \\	
\end{tabular}
} \\
\hline

\textbf{Tech. doc.}: & \\
\hline

\end{tabular}
\end{table}
%-------------------------------------------------------------------------------
\newpage

\subsection{Camera}
%------------------------------------------------------------------------------
\begin{table}[ht!]
\centering
\hspace{-4cm}
\begin{tabular}{|l|l|}
\hline

\textbf{RCT Identifier}: &  RCT-Camera\\
\hline

\textbf{RCT Name}: &  Camara\\
\hline

\textbf{Description}: &  Stereo camera for Higgs' vision system\\
\hline

\textbf{Functionality role}: & Camera \\
\hline

\textbf{Model}: &  Videre STH-MDCS2-C\\
\hline

\textbf{State}: &  available, already controlled independently but not
integrated\\
\hline

\textbf{Serial number}: &  \\
\hline

\textbf{Purchase date}: &  \\
\hline

\textbf{Physical Characteristics}: & 
\parbox{8cm}{
	\begin{tabular}{l l}
		\textbf{Dimensions} & 44x132x73 mm\\	
		\textbf{Weight} & 330 g\\	
	\end{tabular}
}\\
\hline

\textbf{Physical links}: & 
		Screwed to RCT-Wrist\\
\hline

\textbf{Informational links}: & 
\parbox{8cm}{
	\begin{tabular}{l l}
		\textbf{In\&Out} & ieee 1394 (firewire)\\	
	\end{tabular}
}\\
\hline

\textbf{Power}: &  $<$ 1W - 12 VDC\\
\hline

\textbf{Features}: & 
\parbox{10cm}{
	\begin{tabular}{l p{6.5cm}}
		\textbf{Accesories} & RCT-Wrist\\	
	\end{tabular}
}\\
\hline

\textbf{Tech. doc.}: & \\
\hline

\end{tabular}
\end{table}
%-------------------------------------------------------------------------------
\newpage
\subsection{Wrist}
%------------------------------------------------------------------------------
\begin{table}[ht!]
\centering
\hspace{-4cm}
\begin{tabular}{|l|p{9cm}|}
\hline

\textbf{RCT Identifier}: &  RCT-Wrist\\
\hline

\textbf{RCT Name}: &  Wrist \\
\hline

\textbf{Description}: &  2 DOF high precision and torque actuator for
orientation (pan\&tilt) of the RCT-camera\\
\hline

\textbf{Functionality role}: & Wrist \\
\hline

\textbf{Model}: &  SCHUNK PW-070\\
\hline

\textbf{State}: &  Fully integrated \\
\hline

\textbf{Serial number}: &  B3844 \& B3844\\
\hline

\textbf{Purchase date}: &  \\
\hline

\textbf{Physical Characteristics}: &
\parbox{8cm}{
	\begin{tabular}{l l}
    	\multicolumn{2}{l}{}\\
		\textbf{Dimensions} & 165x70x160 mm aprox\\
		\textbf{Weight} & 1.8 kg\\
	\end{tabular}
}\\
\hline

\textbf{Physical links}: &
\parbox{8cm}{
	\begin{tabular}{l l}
		\textbf{axis2 appendix} &  attached to laser support\\
		\textbf{axis1 appendix} &  camera is attached here\\
	\end{tabular}
}\\
\hline

\textbf{Informational links}: &
\parbox{8cm}{
	\begin{tabular}{l l}
		\textbf{In\&Out} & RS-232\\
	\end{tabular}
}\\
\hline

\textbf{Power}: & 24 VDC, nomical power current 4 A per axis \\
\hline

\textbf{Features}: &
\parbox{10cm}{
	\begin{tabular}{l p{6.5cm}}
		\textbf{Sensors} & 1 encoder per axis\\
		\textbf{Nominal torque} & 12 Nm axis1, 2 Nm axis2 \\
		\textbf{Algle of rotation} & 120� axis1, 360� axis2 \\
		\textbf{Resolution} & 5'' axis, 6'' axis2
	\end{tabular} 
}\\
\hline

\textbf{Tech. doc.}: & \\
\hline

\end{tabular}
\end{table}
%-------------------------------------------------------------------------------


\newpage
\subsection{GPS}
%------------------------------------------------------------------------------
\begin{table}[ht!]
\centering
\hspace{-4cm}
\begin{tabular}{|l|l|}
\hline

\textbf{RCT Identifier}: &  RCT-GPS\\
\hline

\textbf{RCT Name}: &  GPS\\
\hline

\textbf{Description}: &  high performance GPS system\\
\hline

\textbf{Functionality role}: & GPSd \\
\hline

\textbf{Model}: &  OEMV-2-RT2\\
\hline

\textbf{State}: &  ready to install\\
\hline

\textbf{Serial number}: &  \\
\hline

\textbf{Purchase date}: &  2007\\
\hline

\textbf{Physical Characteristics}: &
\parbox{8cm}{
	\begin{tabular}{l l}
		\textbf{Dimensions} & 60x100x13 mm\\
		\textbf{Weight} & 56 g\\
	\end{tabular}
}\\
\hline

\textbf{Physical links}: &
\parbox{8cm}{
	\begin{tabular}{l l}
		\textbf{} & \\
	\end{tabular}
}\\
\hline

\textbf{Informational links}: &
\parbox{8cm}{
	\begin{tabular}{l l}
		\textbf{1 USB} & connected to laptop\\
		1 RS-232/422\\
		1 CAN bus & not used\\ 
	\end{tabular}
}\\
\hline

\textbf{Power}: &  1.6W typ. \/ 12 VDC\\
\hline

\textbf{Features}: &
\parbox{10cm}{
	\begin{tabular}{l p{6.5cm}}
		\textbf{Accuracy} & \\
		\textbf{Accesories} & RCT-radio \\
							& Antenna	\\
	\end{tabular}
}\\
\hline

\textbf{Tech. doc.}: & \\
\hline

\end{tabular}
\end{table}
%-------------------------------------------------------------------------------

\subsection{Power Board}
%------------------------------------------------------------------------------
\begin{table}[ht!]
\centering
\hspace{-4cm}
\begin{tabular}{|l|l|}
\hline

\textbf{RCT Identifier}: &  PowerBoard\\
\hline

\textbf{RCT Name}: &  Power Board\\
\hline

\textbf{Description}: &  A simple board for controlling power to other devices\\
\hline

\textbf{Functionality role}: & Power board \\
\hline

\textbf{State}: &  Fully operative\\
\hline

\textbf{Serial number}: &  \\
\hline

\textbf{Manufacturing date}: &  October 2009\\
\hline

\textbf{Physical Characteristics}: &
\parbox{8cm}{
	\begin{tabular}{l l}
		\textbf{Dimensions} & 260x86x16 mm\\
	\end{tabular}
}\\
\hline

\textbf{Physical links}: & Attached to the front of the Vaio support \\
\hline

\textbf{Informational links}: &
\parbox{8cm}{
	\begin{tabular}{l l}
		\textbf{Ribbon cable} & attached to arduino board\\
		\textbf{Input 12V} & connected to Inst. batteries\\
		\textbf{12V-24V} & to connect a 12V to 24V converter\\
		\textbf{Input 24V} & connected to 12V to 24V converter\\
		\textbf{6x12V} & Power for 12V devices\\
		\textbf{3x24V} & Power for Wrist, Laser and 1 free slot\\
		\textbf{Power Jack} & Battery charger\\
	\end{tabular}
}\\
\hline

\textbf{Power}: &  negligible, 12V to 24V \\
\hline


\textbf{Tech. doc.}: & \\
\hline

\end{tabular}
\end{table}
%-------------------------------------------------------------------------------

\newpage
\subsection{Radio}
%-------------------------------------------------------------------------------
\begin{table}[ht!]
\centering
\hspace{-4cm}
\begin{tabular}{|l|p{10cm}|} 
\hline

\textbf{RCT Identifier}: &  RCT-radio\\
\hline

\textbf{RCT Name}: &  radio\\
\hline

\textbf{Description}: &  a radio system for DGPS (GPS + difference with a
known base = more precisssion)\\
\hline

\textbf{Functionality role}: & GPSd radio \\
\hline

\textbf{Model}: &  Pacific Crest PDLRVR\\
\hline

\textbf{State}: &  not installed\\
\hline

\textbf{Serial number}: &  \\
\hline

\textbf{Purchase date}: &  2007\\
\hline

\textbf{Physical Characteristics}: &
\parbox{8cm}{
	\begin{tabular}{l l}
		\textbf{Dimensions} & 21.0 cm L x 6.1 cm diameter\\
	\end{tabular}
}\\
\hline

\textbf{Physical links}: &
\parbox{8cm}{
	\begin{tabular}{l l}
		\textbf{} & \\
	\end{tabular}
}\\
\hline

\textbf{Informational links}: &
\parbox{8cm}{
	\begin{tabular}{l l}
		\textbf{} & \\
	\end{tabular}
}\\
\hline

\textbf{Power}: &  0.3W 9-16 VDC (from its own external battery)\\
\hline

\textbf{Features}: &
\parbox{10cm}{
	\begin{tabular}{l p{6.5cm}}
		\textbf{} & \\
		\textbf{Weight} & 0.34 kg\\
		\textbf{Accessories} & Antenna \\
							 & Battery \\
	\end{tabular}
}\\
\hline

\end{tabular}
\end{table}
%-------------------------------------------------------------------------------

\section{Supporting Systems}

\subsection{PSP Remote Control}
%------------------------------------------------------------------------------
\begin{table}[ht!]
\centering
\hspace{-4cm}
\begin{tabular}{|l|p{10cm}|}
\hline

\textbf{RCT Identifier}: &  RCT-PSP\\
\hline

\textbf{RCT Name}: &  psp\\
\hline

\textbf{Description}: &  a portable device with screen and controls to 
control remotely the RCT platform\\
\hline

\textbf{Functionality role}: & Remote controller \\
\hline

\textbf{Model}: & Sony PSP-1004 \\
\hline

\textbf{State}: &  available, not integrated\\
\hline

\textbf{Serial number}: &  \\
\hline

\textbf{Purchase date}: &  \\
\hline

\textbf{Physical Characteristics}: & 
\parbox{8cm}{
	\begin{tabular}{l l}
		\textbf{Dimensions} & 170x23x74 mm\\	
		\textbf{Weight} & 280 gr\\	
	\end{tabular}
}\\
\hline

\textbf{Physical links}: & none (it's portable) \\
\hline

\textbf{Informational links}: & 
\parbox{8cm}{
	\begin{tabular}{l l}
		\textbf{In\&Out} wifi (IEEE 802.11b), USB, infrared port\\	
	\end{tabular}
}\\
\hline

\textbf{Power}: &  from its internal batteries\\
\hline

\textbf{Features}: & 
\parbox{10cm}{
	\begin{tabular}{l p{6.5cm}}
    	\textbf{Screen} & LCD 4.3'' (16:9) \\
		\textbf{Wifi Security} & WEP, WPA (both AES \& TKIP)\\
		\textbf{Video Compression} & H.264/MPEG-4 AVC \\
	\end{tabular}
}\\
\hline

\end{tabular}
\end{table}
%-------------------------------------------------------------------------------


\newpage
\subsection{Remote Server}
% el servidor de nombres etc

\subsection{Wireless Network}

%------------------------------------------------------------------------------
\begin{table}[ht!]
\centering
\hspace{-4cm}
\begin{tabular}{|l|p{10cm}|}
\hline

\textbf{RCT Identifier}: &  aslab\_wireless \\
\hline

\textbf{RCT Name}: &  aslab\_wireless \\
\hline

\textbf{Description}: &  ASLab's wifi network at DISAM and its surroundings\\
\hline

\textbf{Functionality role}: & Wi-fi connection \\
\hline

\textbf{Access point}: & Linksys WAP54G \\
\hline

\textbf{State}: &  fully operative \\
\hline

\textbf{Serial number}: &  \\
\hline

\textbf{Purchase date}: &  \\
\hline


\textbf{Situation}: & the access point is currently situated in the library \\
\hline

\textbf{Encryption}: & None \\
\hline

\textbf{Features}: & \\
\hline

\textbf{Other}: & config: usr= ; psswd= \\
\hline

\end{tabular}
\end{table}
